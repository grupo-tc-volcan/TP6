\section{Oscilador de Wien}
\subsection{Introducción}
Las señales senoidales son de vital importancia en el ámbito de la electrónica, tanto en usos de intercambio de información como transferencia de energía, entre otros, y 
es justamente por eso que la generación de las mismas es de real importancia.
Una forma de generar señales senoidales mediante componentes analógicos electrónicos es mediante osciladores, y uno de los ejemplos de estos circuitos es el oscilador de 
Wien, cuya forma más elemental se muestra en la Figura \ref{fig:basic_wien_osc}.
\begin{figure}[H]
    \centering
    \includegraphics[width=0.5\textwidth]{../EJ1/Recursos/basic_wien_osc.png}
    \caption{Oscilador de Wien en su forma más sencilla.}
    \label{fig:basic_wien_osc_ex1}
\end{figure}
El objetivo de esta sección del artículo es, basándose en derivaciones del circuito anterior, diseñar un oscilador que genere una señal senoidal con su frecuencia de 
oscilación en $77.5KHz$ como único parámetro de diseño.



\subsection{Marco teórico}
\subsubsection{Frecuencia de oscilación}
Retomando el circuito planteado en la sección anterior, el mismo puede ser analizado para encontrar las relaciones que determinan el los parámetros fundamentales de la 
señal generada por el oscilador.
El mismo puede ser pensado como un amplificador realimentado tanto positiva como negativamente, donde el lazo negativo es el de un amplificador no inversor de ganancia:
\begin{align} 
    & A = 1 + \frac{R_2}{R_1}
    \label{eq:neg_loop_gain_ex1}
\end{align}

La tensión de entrada a este no inversor puede ser puesta en función de los parámetros del lazo negativo mediante la ecuación \ref{eq:v_p_by_v_o_ex1}.
\begin{align}
    & V_p = V_o \frac{R // \frac{1}{j \cdot 2\pi \cdot f \cdot C}}{\left( R // \frac{1}{j \cdot 2\pi \cdot f \cdot C} \right) + \left( R + \frac{1}{j \cdot 2\pi \cdot f \cdot C}\right)}
    \label{eq:v_p_by_v_o_ex1}
\end{align}
Manipulando algebraicamente esta expresión se obtiene la ganancia debida al lazo positivo:
\begin{align}
    & B(jf) = \frac{1}{3 + j \cdot \left(\frac{f}{f_0} - \frac{f_0}{f}\right)}
    \label{eq:pos_loop_gain_ex1}
\end{align}
Donde:
\begin{align}
    & f_0 = \frac{1}{2\pi \cdot R \cdot C}
    \label{eq:osc_freq_ex1}
\end{align}

La ganancia total del oscilador viene dada por la multiplicación de las ganancias de los lazos, consecuentemente, de las ecuaciones \ref{eq:neg_loop_gain_ex1} y \ref{eq:pos_loop_gain_ex1}
se obtiene:
\begin{align}
    & T(jf) = \frac{1 + \frac{R_2}{R_1}}{3 + j \cdot \left(\frac{f}{f_0} - \frac{f_0}{f}\right)}
    \label{eq:total_gain_ex1}
\end{align}

De la observación de esta transferencia puede deducirse que se trata de un pasa banda con ganancia máxima en $f = f_0$ y ganancia:
\begin{align}
    & T_{max} = \frac{1 + \frac{R_2}{R_1}}{3}
    \label{eq:max_gain_ex1}
\end{align}


\subsubsection{Transferencia del circuito}
Este circuito puede ser modelizado como uno realimentado para el cual se cumple que:
\begin{align*}
    & H(s) = \frac{A(s)}{1 + A(s) \cdot \beta(s)}
\end{align*}

Donde para $A(s)$ suficientemente grande, se puede aproximar la expresión a:
\begin{align*}
    & H(s) \approx \frac{1}{\beta(s)}
\end{align*}

Con:
\begin{align*}
    & \beta(s) = \frac{1 + T(s)}{T(s)}
\end{align*}

Y finalmente:
\begin{align}
    & H(s) \approx \frac{\left(\frac{R_2}{R_1} + 1\right) \cdot \left(C^2 \cdot R^2 \cdot s^2 + 3 \cdot C \cdot R \cdot s + 1\right)}
    {C^2 \cdot R^2 \cdot s^2 + \left(\frac{R_2}{R_1} + 4\right) \cdot C \cdot R \cdot s + 1}
    \label{eq:transference_ex1}
\end{align}


\subsubsection{Criterio de Barkhausen}
El criterio de Barkhausen dicta las condiciones que deben cumplirse en oscilador como el de Wien para que se cumpla la condición de oscilación y determina la frecuencia a 
la que se dará el mismo.
El criterio consiste en hallar el punto para el cual se cumple que la ganancia es 1 y el desfasaje es nulo, y, para el circuito en cuestión, observamos que la segunda de 
las condiciones puede ser cumplida evaluando en $f_0$, mientras que la primera puede cumplirse para esa misma frecuencia si $\frac{R_2}{R_1} = 2$


\subsubsection{Desviaciones del circuito básico}
Si bien el circuito previamente mencionado puede ser utilizado para buscar de forma más sencilla las relaciones que determinan el cumplimiento o no de la oscilación, en 
la práctica deben de hacerse modificaciones que prevengan al circuito de las variaciones intrínsecas a la utilización de componentes reales.
Como se dijo en la sección anterior, para que se cumpla el criterio de Barkhausen, debe darse que $\frac{R_2}{R_1} = 2$ exactamente; sin embargo, esta característica no 
puede ser lograda en la realidad con simples componentes pasivos, razón por la cual el circuito original de la Figura \ref{fig:basic_wien_osc_ex1}, es reemplazado por el 
de la Figura \ref{fig:wein_osc_circuit_ex1}.
\begin{figure}[H]
    \centering
    \includegraphics[width=0.5\textwidth]{../EJ1/Recursos/wein_osc_circuit.jpg}
    \caption{Circuito utilizado.}
    \label{fig:wein_osc_circuit_ex1}    
\end{figure}

\textbf{Transistor JFET}\\
La principal diferencia yace en la inclusión de un JFET-N en la realimentación negativa, en serie con $R_1$, cuya función es la de resistencia variable 
dinámicamente.
Mediante esta incorporación, puede colocarse una resistencia en $R_1$ de valor tal que $\frac{R_2}{R_1} > 2$ que en el encendido del circuito favorezca 
la realimentación positiva, y se inicie la oscilación.
Una vez llegado al valor deseado de tensión de salida de la señal (fijado por los diodos, y explicado en el siguiente ítem), la tensión en el Gate del 
transistor se modificará achicando el ancho del canal y aumentando entonces la resistencia del mismo.
Con este mecanismo es que se logra que la ganancia del lazo negativo llegue a la condición de $\frac{R_2}{R_1} = 2$.

La elección de qué tipo de JFET se usa va de la mano de la disposición de los diodos.
Cabe recordar que un JFET canal N disminuirá el ancho de su canal conforme se aplique tensión negativa en su Gate, mientras que el canal P lo hará ante 
tensiones de valor positivo.
Consecuentemente, el uso de un canal N realizará un control de la oscilación a través del semiciclo negativo de la misma, mientras que un canal P lo hará 
para el positivo.

Otro aspecto a considerar es la zona de utilización del transistor y el rango de variación de la tensión en el Gate.
Este JFET está siendo usado en la zona lineal, con bajas tensiones en el Gate, empleándose como resistencia variable.
En cuanto a la variación de estas tensiones, se pretende que sean chicas, ya que la no linealidad en la respuesta del transistor podría introducir 
distorsiones en caso de que no lo fueran.
\newline

\textbf{Diodo rectificador y Zener ($D_R$ y $D_Z$)}
Su función, como bien se mencionó antes, es la de limitar la amplitud de la señal generada a la salida.
Cada vez que la salida del operacional supera el valor dado por $V_Z + V_D$, más la tensión en el capacitor cargado, los diodos limitan la amplitud a la 
salida.
En el circuito de la Figura \ref{fig:wein_osc_circuit_ex1}, los diodos están dispuestos de forma tal que la limitación de la amplitud se dé en el 
semiciclo negativo, dado que se está utilizando un JFET canal N para el control de la ganancia del lazo negativo.
En caso de querer usar un JFET canal P, ambos deberían ser invertidos en su conexión, para permitir el control en el semiciclo positivo.

En caso de querer modificar la amplitud a la salida de la señal, hay dos caminos que pueden tomarse.
El primero y más sencillo es elegir convenientemente el diodo Zener a utilizar; sin embargo, si se requiere un valor de salida que no pueda ser satisfecho 
mediante valores comerciales de los Zener, puede colocarse una resistencia en la rama de los diodos que, al momento en que los diodos permitan el paso de 
la corriente, caerá una tensión sobre ella que permitirá el aumento de la amplitud en la senoidal.
\newline

\textbf{Capacitor y resistencia $R_x$ y $C_x$} \label{sec:r_x_c_x_ex1}
Esta malla RC, la cual puede ser vista como un pasa-bajos, cumple la función de mantener la polarización del transistor JFET durante todo el ciclo de la 
senoidal.
Se busca que la constante de tiempo del mismo sea alta, para lograr así el deseado efecto de que la variación en la tensión de Gate sea baja, y esto se 
logra colocando valores altos de $R_x$.



\subsection{Elección de componentes}
\subsubsection{Lazo de realimentación positiva}
En cumplimiento con el parámetro de diseño de $f_0 = 77.5KHz$, y dados los dos grados de libertad ($C$ y $R$) de los que se dispone para el diseño de un 
oscilador que cumpla con lo pedido, se fija en primer lugar $C = 1nF$, ya que será más sencillo ajustar un valor de resistencia que de un capacitor.
Como consecuencia de esto, y haciendo uso de la ecuación \ref{eq:osc_freq_ex1}, se obtiene que:
\begin{align*}
    & R = \frac{1}{2\pi \cdot C \cdot f_0} \approx 2.05K\Omega
\end{align*}

Dado que no se cuenta con un valor comercial para esa resistencia, y teniendo en cuenta la importancia crítica de este parámetro, se decide colocar un 
preset para calibrar este aspecto post-diseño.


\subsubsection{Lazo de realimentación negativa}
Para la determinación de los valores de resistencias $R_1$ y $R_2$, se retoma lo dicho en el marco teórico sobre la necesidad de que, inicialmente, 
$\frac{R_2}{R_1} > 2$.
Nuevamente se tienen dos grados de libertad, de modo que se fija $R_2 = 100K\Omega$, y se busca $R_1 < 50K\Omega$, encontrando como valor comercial más 
conveniente $R_1 = 47K\Omega$.
En este caso también será utilizado un preset para modificar la resistencia $R_1$ hasta la condición buscada.

Para el Control Automático de Ganancia (CAG), compuesto por el diodo rectificador, el diodo Zener, el transistor y el pasa-bajos de $C_x$ y $R_x$, 
se seleccionan los componentes 1N4148, 1N5230, MPF102, $10\mu F$ y $1M\Omega$, respectivamente.
Los últimos dos son elegidos respetando el criterio de la sección \ref{sec:r_x_c_x_ex1}, con la intensión de que la descarga del capacitor sea lo 
suficientemente lenta


\subsubsection{Selección del operacional a utilizar}
Para la selección del amplificador operacional al utilizar se tienen en cuenta diferentes aspectos característicos de los mismos que hacen a la 
performance del circuito en el que serán empleados.

Algunas de estas características importantes son la tensión de saturación, slew rate y cross-over, ya que estos aspectos pueden tener implicancia en la 
distorsión armónica de la señal en la salida.
Además, ha de observarse también el GBP del amplificador, ya que la presencia de polos que distorsionen la fase en frecuencias cercanas a la de diseño 
podría provocar problemas para el cumplimiento del criterio de Barkhausen.

La información recopilada de hojas de datos es comprimida en la tabla \ref{tab:opamp_comparison_ex1}, de la cual se analizan como buenas opciones a los 
operacionales TL082, TL072 y LM833.
Se considera que cualquiera de estas tres opciones satisface los requerimientos de este circuito, razón por la cual la decisión final de elegir al TL082 
se basa en buenas experiencias previas con el uso de este amplificador.
\begin{table}[H]
    \centering
    \begin{tabular}{c|ccccccc}
    \textbf{Operacional} & \textbf{$A_{vol}$} & \textbf{$GBP$} & \textbf{$Z_{in}$} & \textbf{$Z_{out}$} & \textbf{$SR$}         & \textbf{$THD$} & \textbf{$V_{CC_{max}}$} \\ \hline
    \textbf{TL082}       & $200 \cdot 10^3$   & $3MHz$         & $10^12\Omega$     & $-$                & $13 \frac{V}{\mu s}$  & $< 0.0003$     & $18V$                   \\
    \textbf{TL072}       & $200 \cdot 10^3$   & $3MHz$         & $10^12\Omega$     & $-$                & $13 \frac{V}{\mu s}$  & $< 0.0003$     & $36V$                   \\
    \textbf{LM833}       & $316 \cdot 10^3$   & $16MHz$        & $2^12\Omega$      & $37\Omega$         & $13 \frac{V}{\mu s}$  & $< 0.0002$     & $36V$                   \\
    \textbf{LM324}       & $100 \cdot 10^3$   & $1.2MHz$       & $2^6\Omega$       & $-$                & $0.5 \frac{V}{\mu s}$ & $-$            & $32V$                   \\
    \textbf{LM741}       & $200 \cdot 10^3$   & $1.5MHz$       & $2^6\Omega$       & $75\Omega$         & $13 \frac{V}{\mu s}$  & $0.0006$       & $22V$                  
    \end{tabular}
    \caption{Comparación de operacionales.}
    \label{tab:opamp_comparison_ex1}
\end{table}



\subsection{Caracterización del sistema}
\subsubsection{Singularidades}
La posición de las singularidades en el plano complejo es de vital importancia para estos circuitos osciladores, es por eso que una correcta caracterización 
del mismo requiere de un análisis de singularidades.

En el presente artículo, este análisis se realizará sometiéndolo a variaciones en la resistencia variable lograda con el transistor JFET, y mediante el 
auxilio de software de cálculo como los módulos de lenguaje Python, scipy.
Se hizo variar tal resistencia entre los valores de $0$ a $5K\Omega$ con un paso de $100\Omega$, y el resultado es el ilustrado en la Figura \ref{fig:closed_loop_poles_ex1}.
\begin{figure}[H]
    \centering
    \includegraphics[width=0.8\textwidth]{../EJ1/Recursos/closed_loop_poles.png}
    \caption{Polos del lazo cerrado.}
    \label{fig:closed_loop_poles_ex1}    
\end{figure}

Puede apreciarse como para todos los valores tomados, los polos se encuentran encimados sobre el eje imaginario, condición que permitirá la oscilación.

Por otro lado, haciendo uso de la transferencia del sistema expresada en la ecuación \ref{eq:transference_ex1}, y aplicando la misma variación del parámetro 
de la resistencia del JFET, se obtiene el diagrama de la Figura \ref{fig:transference_poles_and_zeros_ex1}.
\begin{figure}[H]
    \centering
    \includegraphics[width=\textwidth]{../EJ1/Recursos/transference_poles_and_zeros.png}
    \caption{Polos del lazo cerrado.}
    \label{fig:transference_poles_and_zeros_ex1}    
\end{figure}


\subsubsection{Sensibilidades}
En el análisis de sensibilidades, el parámetro de interés en este trabajo es la frecuencia de oscilación, y será por lo tanto el parámetro al cual se 
debe prestar atención y establecer su dependencia de los valores de los componentes del diseño.

De la ecuación \ref{eq:osc_freq_ex1} puede observarse que las sensibilidades de la frecuencia de oscilación con respecto a $R$ y $C$ (únicos componentes 
de los que depende), son:
\begin{align*}
    & S_C^{f_0} = S_R^{f_0} = -1
\end{align*}



\subsection{Simulación}
Mediante herramientas de simulación como LTspice, se procede a simular el circuito propuesto para ver algunas de sus características y luego compararlas 
con los resultados medidos.

Tal como se explicó en secciones previas, existe un tiempo de establecimiento de la señal para el cual la realimentación positiva amplifica gradualmente 
la señal a la salida hasta llegar a la amplitud estacionaria.
Este tiempo puede ser observado mediante la lectura del gráfico \ref{fig:time_ex1}, que 
indica que se se puede considerar la señal en estado estacionario una vez transcurridos los $10ms$.
\begin{figure}[H]
    \centering
    \includegraphics[width=0.8\textwidth]{../EJ1/Recursos/time.png}
    \caption{Establecimiento de la señal simulada.}
    \label{fig:time_ex1}    
\end{figure}

Además del tiempo de establecimiento, es interesante observar desde la simulación el espectro de la señal generada, el cual es de utilidad para determinar 
la distorsión armónica total (THD por su sigla en inglés).
La misma es calculada mediante el cociente $\frac{P_{armonicos}}{P_{total}}$, y nos da una idea de la potencia perdida de la señal en 
frecuencias armónicas que no son de nuestro interés.
En esta ecuación, $P_{armonicos}$ es la potencia de los armónicos, mientras que $P_{total}$ es además con la potencia de la senoidal principal sumada.

El espectro de la señal es el que se ilustra en la Figura \ref{fig:fft_ex1} y del cual se extraen los valores de los principales armónicos de la tabla \ref{tab:thd_ex1}
\begin{figure}[H]
    \centering
    \includegraphics[width=0.8\textwidth]{../EJ1/Recursos/fft.png}
    \caption{Establecimiento de la señal simulada.}
    \label{fig:fft_ex1}    
\end{figure}

\begin{table}[H]
    \centering
    \begin{tabular}{c}
    \textbf{Potencia (dB)} \\ \hline
    $11$                   \\
    $-23.9$                \\
    $-32.3$                \\
    $-42.5$                \\
    $-41.8$                \\
    $-54.1$                \\
    $-48.5$                \\
    $-49.7$               
    \end{tabular}
    \caption{Potencia de los primero armónicos.}
    \label{tab:thd_ex1}
\end{table}

Con los valores de la tabla, y haciendo uso de la ecuación del THD, se obtiene que:
\begin{align*}
    & THD = 0.0308
\end{align*}



\subsection{Resultados y análisis}
Se implementa el diseño en un PCB, el cual es luego sometido a mediciones para corroborar que cumpla con los requerimientos para los cuales fue diseñado.
A continuación se detallan las mediciones que se realizaron sobre el circuito y las conclusiones a las que se llegó.


\subsubsection{Frecuencia y amplitud de oscilación}
La frecuencia de oscilación es el principal parámetro de diseño y al cual se le prestó mayor atención a la hora de que se cumpliese, obteniéndose resultados 
satisfactorios luego de una calibración fina de las resistencias utilizadas mediante presets.

La Figura \ref{fig:actual_freq_ex1} representa la señal obtenida en régimen permanente, de donde se puede apreciar de las mismas mediciones del osciloscopio 
como la frecuencia lograda es la buscada.
No es ese el caso para la amplitud, donde se esperaba una salida por encima de los $10.8 V_{pp}$, derivados de la suma de las tensiones del Zener y del 
diodo rectificador.
En cambio, la salida medida es de $8.94V_{pp}$, y se conjetura que este valor por debajo de lo esperado puede deberse a variaciones en la tensión del Zener.
\begin{figure}[H]
    \centering
    \includegraphics[width=0.8\textwidth]{../EJ1/Recursos/actual_freq.png}
    \caption{Oscilación en régimen permanente con la frecuencia de diseño.}
    \label{fig:actual_freq_ex1}    
\end{figure}

Variando el preset utilizado para el circuito se lo llevo a las máximas y mínimas frecuencias a las que se pudo hacer oscilar al circuito, y las mismas fueron 
de $68.5KHz$ a $93KHz$, mediciones que pueden ser corroboradas mediante la Figura \ref{fig:min_max_freq_ex1}.
\begin{figure}[H]
    \centering
    \begin{tabular}{c}
        \includegraphics[width=0.7\textwidth]{../EJ1/Recursos/min_freq.png} \\
        \includegraphics[width=0.7\textwidth]{../EJ1/Recursos/max_freq.png}
    \end{tabular}
    \caption{Mínima y máxima frecuencia de oscilación del circuito.}
    \label{fig:min_max_freq_ex1}    
\end{figure}

Y finalmente se estudió el rango de amplitud de señal de salida que podía ofrecer este oscilador, el cual, además de ser dependiente de los diodos, lo es 
de la tensión de alimentación del operacional, que cabe recordar que puede ser alimentado hasta con $18V_{pp}$ según recomendaciones de su datasheet.

Se buscó la amplitud del circuito en esta condición límite, para la frecuencia de diseño, así como para la mínima alimentación con la cual se apreciaba 
oscilación a la salida, que fue con $V_{CC} = \pm3V$.
Los resultados encontrados son los de la Figura \ref{fig:min_max_amp_ex1}, con $1.25V_{pp}$ de mínima y $10.5V_{pp}$ de máxima.
\begin{figure}[H]
    \centering
    \begin{tabular}{c}
        \includegraphics[width=0.7\textwidth]{../EJ1/Recursos/min_amp.png} \\
        \includegraphics[width=0.7\textwidth]{../EJ1/Recursos/max_amp.png}
    \end{tabular}
    \caption{Mínima y máxima amplitud de oscilación del circuito.}
    \label{fig:min_max_amp_ex1}    
\end{figure}


\subsubsection{Establecimiento de la señal}
Se observó también mediante el modo de Roll del osciloscopio el establecimiento de la señal, y lo medido es lo que se muestra en la Figura \ref{fig:start_up_ex1}.
\begin{figure}[H]
    \centering
    \includegraphics[width=0.8\textwidth]{../EJ1/Recursos/start_up.png}
    \begin{tabular}{c c}
        \includegraphics[width=0.7\textwidth]{../EJ1/Recursos/start_up_zoom_x1.png}
        \includegraphics[width=0.7\textwidth]{../EJ1/Recursos/start_up_zoom_x2.png}
    \end{tabular}
    \caption{Establecimiento de la señal medido.}
    \label{fig:start_up_ex1}    
\end{figure}
Se desprende del análisis de la figura que el tiempo de establecimiento ronda los $8ms$, en concordancia y hasta mejorando lo esperado desde la simulación.



\subsection{Conclusión}
A modo de cierre d esta sección se repasan los resultados obtenidos del diseño e implementación de un oscilador de Wien, cuyo parámetro de diseño es su 
frecuencia de oscilación.
Es justamente en esta restricción la que mejor se logró cumplir, ya que el ajuste del circuito mediante presets permitió una calibración precisa para acomodar 
la frecuencia de oscilación al valor buscado.
Resultaron también positivas las mediciones de tiempo de establecimiento, siendo del órden o incluso más rápido de lo esperado según simulación.
Sin embargo, y por último, la amplitud de la oscilación se ubicó por debajo de los valores indicados, siendo esta la característica de menor desempeño de la 
implementación.